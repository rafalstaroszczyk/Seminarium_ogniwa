\documentclass[xcolor]{beamer}
\usepackage[utf8]{inputenc}
\usepackage[T1]{fontenc}
\usepackage[polish]{babel}
\usepackage{csquotes}
\usepackage[sorting=none,giveninits=true]{biblatex}
\addbibresource{bibliography.bib}
\usepackage{float}
\usepackage{xcolor}  % kolory motywu
\usepackage{tikz}
\usetikzlibrary{angles}
\usetikzlibrary{quotes}
\usetikzlibrary{decorations.pathreplacing}
\usetikzlibrary{calligraphy}
\usetikzlibrary{arrows.meta}
\usetikzlibrary{calc}
\usepackage{pgfplots}
\usepackage{pgfplotstable}
\pgfplotsset{compat=1.9}
\usepackage{amsmath}  % równania
\usepackage{amssymb}
\usepackage{bbold}
\usepackage{physics2}  % pochodne, macierze itp
\usephysicsmodule{ab}
\usephysicsmodule{diagmat}
\usephysicsmodule{xmat}
\usephysicsmodule{nabla.legacy}
\usephysicsmodule{op.legacy}
\usefonttheme[onlymath]{serif}
\makeatletter
%\newcommand\vb[1]{\@ifstar\boldsymbol\mathbf{#1}}
\newcommand\vb[1]{\@ifstar\boldsymbol\mathbf{#1}}
\newcommand\va[1]{\@ifstar{\vec{#1}}{\vec{\mathrm{#1}}}}
\newcommand\vu[1]{%
\@ifstar{\hat{\boldsymbol{#1}}}{\hat{\boldsymbol{#1}}}}
\makeatother
\usepackage{fixdif, derivative}  % pochodne
\usepackage[version=4]{mhchem}
\title{Perowskitowe ogniwa słoneczne - na jakim etapie technologicznym jesteśmy?}
\author{Rafał Staroszczyk}
\date{}
\usetheme{Hannover}
\usecolortheme{spruce}

\DeclareMathOperator{\Col}{Col}
\DeclareMathOperator{\Nul}{Nul}
\DeclareMathOperator{\arctg}{arctg}
\DeclareMathOperator{\tgh}{tgh}


\setlength{\abovedisplayskip}{0pt}
\setlength{\belowdisplayskip}{0pt}
\setlength{\abovedisplayshortskip}{0pt}
\setlength{\belowdisplayshortskip}{0pt}

\newcommand{\inv}[1]{\frac{1}{#1}}

\begin{document}
\maketitle
\begin{frame}{Spis treści}
\tableofcontents
\end{frame}

\section{Struktura perowskitu}
\begin{frame}{Struktura perowskitu}
\begin{columns}
\begin{column}{0.5\textwidth}
\begin{figure}[H]
\centering
\includegraphics[width=\textwidth]{"perovskite".png}
\caption{Ogólna struktura perowskitu: \ce{ABX3}, \cite{optoelectronic_devices}}
\end{figure}
\end{column}

\begin{column}{0.5\textwidth}
\begin{block}{}
Pierwiastek \ce{A} umieszczony jest w wierzchołkach \\
Pierwiastek \ce{X} umieszczony jest w środkach ścian \\
Pierwiastek \ce{B} umieszczony jest w środku sześcianu
\end{block}
\begin{block}{Różne jony}
\begin{tabular}{rl}
1:2 & \ce{A^{+}B^{2+}X^{-}3} \\
2:4 & \ce{A^{2+}B^{4+}X^{2-}3} \\
3:3 & \ce{A^{3+}B^{3+}X^{2-}3} \\
1:5 & \ce{A^{+}B^{5+}X^{2-}3}
\end{tabular}
\end{block}
\end{column}
\end{columns}
\end{frame}

\section{Zastosowanie perowskitów}
\begin{frame}{Zastosowanie perowskitów}
\begin{block}{Zastosowania \cite{optoelectronic_devices}}
\begin{enumerate}
\item Fotodioda
\item Dioda elektroluminescencyjna
\item Laser
\item Ogniwo fotowoltaiczne
\item Memrystor
\item Pressure-induced emission devices
\end{enumerate}
\end{block}
\end{frame}

\section{Perowskity halogenkowe}
\subsection{Stabilność strukturalna perowskitów 3D}
\begin{frame}{Stabilność strukturalna}
\begin{block}{Goldschmidt tolerance factor}
\begin{equation}
\tau = \frac{r_A + r_X}{\sqrt{2}\pab{r_B + r_X}}
\end{equation}
\end{block}
\begin{figure}[H]
\centering
\includegraphics[width=0.5\textwidth]{"goldschmidt".png}
\caption{Tolerance factor dla wybranych jonów \ce{A} w \ce{APbI3} \cite{miyasaka}}
\end{figure}
\end{frame}

\section{Perowskit quasi-2D}
\begin{frame}{Perowskit quasi-2D \cite{ding}}
%\begin{block}{Dlaczego}
%Perowskity quasi-2D wykazują wyższą odporność na warunki zewnętrzne
%\end{block}
\begin{block}{Struktury}
Może występować w kilku strukturach:
\begin{enumerate}
\item Ruddlesden-Popper (R-P)
\item Dion-Jacobson (D-J)
\item Alternating Cation in Interlayer space (ACI)
\item Aurivillius (AV)
\end{enumerate}
\end{block}
\begin{figure}[H]
\centering
\includegraphics[width=0.5\textwidth]{"3d2d".png}
\caption{Perowskity 3D oraz 2D \cite{optoelectronic_devices}}
\end{figure}
\end{frame}

\subsection{Ruddlesden-Popper (R-P)}
\begin{frame}{Ruddlesden-Popper (R-P)}
\begin{block}{Wzór}
\ce{{A'}^{+}2 A^{+}_{n-1} B^{2+}_{n} X^{-}_{3n+1}}
\end{block}
\begin{block}{Przykładowe kationy}
\begin{enumerate}
\item Butyloamina (BA; \ce{CH3(CH2)3NH3^+})
\item Fenyloetyloamina (PEA; \ce{Ph(CH2)2NH3^+})
\end{enumerate}
\end{block}
\end{frame}

\subsection{Dion-Jacobson (D-J)}
\begin{frame}{Dion-Jacobson (D-J)}
\begin{block}{Wzór}
\ce{{A'}^{2+} A^{+}_{n-1} B^{2+}_{n} X^{-}_{3n+1}}
\end{block}
\begin{block}{Przykładowe kationy}
\begin{enumerate}
\item Propylodiamina (PDA; \ce{NH3^+(CH2)3NH3^+})
\item 3-(aminomethyl)pyridine (3AMP)
\end{enumerate}
\end{block}
\end{frame}

\subsection{Alternating Cation in Interlayer space (ACI)}
\begin{frame}{Alternating Cation in Interlayer space (ACI)}
\begin{block}{Wzór}
\ce{{A'}^{+} A^{+}_{n} B^{2+}_{n} X^{-}_{3n+1}}
\end{block}
\begin{block}{Kationy}
\begin{enumerate}
\item Guanidinium (GA; \ce{C(NH2)^{+}3})
\end{enumerate}
\end{block}
\end{frame}

\subsection{Ogniwo quasi-2D}
\begin{frame}{Ogniwo quasi-2D}
\begin{figure}[H]
\centering
\includegraphics[width=\textwidth]{"ogniwo2d".png}
\caption{Ogniwo na bazie perowskitu quasi-2D \cite{ding}}
\end{figure}
\end{frame}

\begin{frame}
\begin{figure}[H]
\centering
\includegraphics[width=\textwidth]{"degradacja2d".png}
\caption{Stabilność różnych konfiguracji \cite{ding}}
\end{figure}
\begin{figure}[H]
\centering
\includegraphics[width=0.8\textwidth]{"strukturaRPDJ".png}
\caption{Badane struktury R-P oraz D-J \cite{ding}}
\end{figure}
\end{frame}

\subsection{Ogniwo hybrydowe 2D/3D}
\begin{frame}{Ogniwo hybrydowe 2D/3D}
Poprzez zmieszanie roztworu perowskitów 2D i 3D można otrzymać strukturę hybrydową. Stabilizuje to perowskity \ce{FA}.
\begin{figure}[H]
\centering
\includegraphics[width=\textwidth]{"ogniwohybr2d3d".png}
\caption{Ogniwo hybrydowe 2D/3D \cite{ding}}
\end{figure}
\end{frame}

\section{Perowskity halogenkowe nieorganiczne}
\begin{frame}{Perowskity halogenkowe nieorganiczne}
Najważniejszym perowskitem w tej grupie jest \ce{CsPbI3}, lecz jest on niestabilny. Można to poprawić na kilka sposobów.
\begin{figure}[H]
\centering
\includegraphics[width=\textwidth]{"fazycspbi3".png}
\caption{Fazy \ce{CsPbI3} \cite{miyasaka}}
\end{figure}
\end{frame}

\subsection{Podmiana X}
\begin{frame}{Podmiana X \cite{miyasaka}}
Perowskity \ce{CsPbCl3} oraz \ce{CsPbBr3} są stabilniejsze, ale mają wyższą przerwę energetyczną, przez co mniej się nadają w zastosowaniu do ogniw fotowoltaicznych. Można spróbować częściowego podstawienia.
\begin{center}
\ce{CsPbI_{3-x}Br_{x}}
\end{center}
Podwyższona wartość przerwy energetycznej perowskitów takich jak \ce{CsPbI2Br} oraz zwiększona stabilność może być wykorzystana w ogniwach tandemowych.
\end{frame}

\subsection{Podmiana B}
\begin{frame}{Podmiana B \cite{miyasaka, ding}}
Alternatywą jest podstawienie innego metalu w miejsce \ce{Pb}. Ma to dodatkową zaletę zmniejszenia ilości ołowiu. Przykładowo \ce{Sn^{2+}}; utlenia się do \ce{Sn^{4+}}, ale tworzy \ce{Cs2SnI6}.
\begin{block}{Podwójny perowskit}
\ce{Cs2{M'}^{+}{M''}^{3+}X6}
\end{block}
Teoretycznie \ce{Cs2AgInBr6} może osiągnąć wydajność $28\%$. W praktyce sprawność nie przekracza kilku procent.
\end{frame}

\subsection{Podmiana A}
\begin{frame}{Podmiana A \cite{miyasaka}}
Podmiana części \ce{Cs} na inny metal może zwiększyć stabilność i wydajność. Przykładowo \ce{Cs_{0.99}Rb_{0.01}PbI2Br} osiągnął rekordową sprawność $17,16\%$.
\end{frame}

\begin{frame}{Bibliography}
\printbibliography
\end{frame}
\end{document}